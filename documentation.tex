\documentclass[a4paper,12pt]{article}
\usepackage[utf8]{inputenc}
\usepackage[T2A]{fontenc}
\usepackage[uzbek]{babel}
\usepackage{amsmath}
\usepackage{graphicx}
\usepackage{geometry}
\geometry{a4paper, margin=1in}
\usepackage{hyperref}
\usepackage{xcolor}

% Defining document structure and packages
% Configuring page layout and fonts
\usepackage{DejaVuSans} % Reliable font for Uzbek
\usepackage{amsfonts}

\title{Ipak Yo'li Bank assistant Hujjati}
\author{Neo}
\date{\05.05.2025}
\label{Epsilon AI Agent}

\begin{document}

\maketitle

\section{Loyiha haqida}
% Describing the project purpose and functionality
Bu hujjat Ipak Yo'li banki uchun ishlab chiqilgan chatbot/voice assistantning texnik tavsifini o'z ichiga oladi. Chatbot/voice assistant foydalanuvchilarning ovozli va matnli so'rovlarini qabul qilib, kredit limitlarini bashorat qilish va bank ma'lumotlarini taqdim etish uchun mo'ljallangan.

\section{O'rnatish va ishlatish}
% Providing installation and usage instructions
\subsection{O'rnatish}
Python 3.13.3+ ni o'rnatish zarur. Quyidagi kutubxonlarni o'rnatish kerak:
\begin{itemize}
    \item pandas
    \item joblib
    \item requests
    \item google-generativeai
    \item scipy
    \item sounddevice
    \item pygame
    \item together
    \item dotenv
\end{itemize}
`.env` faylida API kalitlarini belgilang:
\begin{itemize}
    \item GEMINI\_API\_KEY
    \item TTS\_API\_KEY
    \item TOGETHER\_API\_KEY
\end{itemize}
Fayllarni yuklang: `linear_regression_model.pkl`, `test_data2.csv`, `general_info.txt`.

\subsection{Ishlatish}
Dasturni `python terminal.py` bilan ishga tushiring. "o" (ovoz) yoki "m" (matn) tanloving. Ovozli xabar uchun 7 soniya ichida gapiring. "exit" bilan chiqish.

\section{Funksiyalar}
% Listing main features
\begin{itemize}
    \item Ovozli va matnli kiritish qo'llab-quvvatlanadi.
    \item Kredit limiti ID asosida bashorat qilinadi.
    \item Bank haqida umumiy ma'lumotlar taqdim etiladi.
    \item O'zbekcha (Krill va Lotin) tillarini qo'llab-quvvatlaydi.
\end{itemize}

\section{Muallif va aloqa}
% Providing author and contact information
Loyiha Neo tomonidan ishlab chiqilgan.

\end{document}
